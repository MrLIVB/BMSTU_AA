\chapter{Технологический раздел}
\label{cha:impl}
В данном разделе будут представлены листинги кода реализованных алгоритмов, указаны использованные оптимизации алгоритма Винограда.
\section{Средства реализации}
В данной работе используется язык программирования С++. Среда разработки Visual Studio Code. Для замера процессорного времени используется функция QueryPerformanceCounter из библиотеки windows.h.
\section{Листинг кода}
В листингах 3.1--3.3 приведены стандартный алгоритм умножения матриц, алгоритм умножения матриц Винограда и оптимизированный алгоритм Винограда.
\begin{lstlisting}[caption= Стандартный алгоритм умножения матриц]
void multiplyMatricesStandard(int** matrix1, int m, int n, int** matrix2, int q, int** result)
{
	for (int i = 0; i < m; i++)
	{
		for (int j = 0; j < q; j++)
		{
			result[i][j] = 0;
			for (int k = 0; k < n; k++)
			{
				result[i][j] = result[i][j] + matrix1[i][k] * matrix2[k][j];
			}
		}
	}
}
\end{lstlisting}

\begin{lstlisting}[caption= Алгоритм умножения матриц Винограда]
void multiplyMatricesWinograd(int **matrix1, int m, int n, int **matrix2, int q, int **result)
{
	int *mulH = new int[m];
	for (int i = 0; i < m; i++)
	{
		mulH[i] = 0;
		for (int k = 0; k < n / 2; k++)
			mulH[i] = mulH[i] + matrix1[i][2*k] * matrix1[i][2*k + 1];
	}
	
	int *mulV = new int[q];
	for (int i = 0; i < q; i++)
	{
		mulV[i] = 0;
		for (int k = 0; k < n / 2; k++)
			mulV[i] = mulV[i] + matrix2[2*k][i] * matrix2[2*k+1][i];
	}
	
	for (int i = 0; i < m; i++)
		{
		for (int j = 0; j < q; j++)
		{
			result[i][j] = -mulH[i] - mulV[j];
			for (int k = 0; k < n/2; k++)
				result[i][j] = result[i][j] + (matrix1[i][2 * k] + matrix2[2 * k + 1][j])*(matrix1[i][2 * k + 1] + matrix2[2 * k][j]);
		}
	}
	
	if (n % 2 == 1)
	{
		for (int i = 0; i < m; i++)
			for (int j = 0; j < q; j++)
				result[i][j] = result[i][j] + matrix1[i][n - 1] * matrix2[n-1][j];
	}
	delete[] mulH;
	delete[] mulV;
}
\end{lstlisting}

\begin{lstlisting}[caption= Улучшенный алгоритм матриц Винограда]
void multiplyMatricesWinogradOptimized(int **matrix1, int m, int n, int **matrix2, int q, int **result)
{
	int buf;
	const int isOdd = n % 2;
	const int t = n - 1;
	if (isOdd)
		n -= 1;
	
	
	int *mulH = new int[m];
	for (int i = 0; i < m; i++)
	{
		buf = 0;
		for (int k = 0; k < n; k += 2)
			buf -= matrix1[i][k] * matrix1[i][k + 1];
		mulH[i] = buf;
	}
	
	int *mulV = new int[q];
	for (int i = 0; i < q; i++)
	{
		buf = 0;
		for (int k = 0; k < n; k += 2)
			buf -= matrix2[k][i] * matrix2[k+1][i];
		mulV[i] = buf;
	}
	
	for (int i = 0; i < m; i++)
	{
		for (int j = 0; j < q; j++)
		{
			buf = mulH[i] + mulV[j];
			for (int k = 0; k < n; k += 2)
				buf += (matrix1[i][k] + matrix2[k + 1][j]) * (matrix1[i][k + 1] + matrix2[k][j]);
			if (isOdd==1)
				buf += matrix1[i][t] * matrix2[t][j];
			result[i][j] = buf;
		}
	}
	delete[] mulH;
	delete[] mulV;  
}
\end{lstlisting}

\subsection{Оптимизация алгоритма Винограда}
\begin{enumerate}[1.]
	\item Заранее вычисляются такие константы как isOdd=n\%2 и t=n-1.
	\item Накопление результата в буфер, а после помещение буфера в ячейку.
	\item Где это возможно используются -= и +=.
	\item Вместо n/2 цикл идёт до n с шагом 2, что представлено в листинге 3.4.
	\begin{lstlisting}[caption=Оптимизации алгоритма Винограда \textnumero 2--4]
	for (int i = 0; i < m; i++)
	{
		buf = 0;
		for (int k = 0; k < n; k += 2)
			buf -= matrix1[i][k] * matrix1[i][k + 1];
		mulH[i] = buf;
	}
	\end{lstlisting}
	\item Объединены части 3 и 4 алгоритма Винограда, как это отображено в листинге 3.5
	\begin{lstlisting}[caption=Оптимизация алгоритма Винограда \textnumero 5]
	for (int i = 0; i < m; i++)
	{
		for (int j = 0; j < q; j++)
		{
			buf = mulH[i] + mulV[j];
			for (int k = 0; k < n; k += 2)
				buf += (matrix1[i][k] + matrix2[k + 1][j]) * (matrix1[i][k + 1] + matrix2[k][j]);
			if (isOdd==1)
				buf += matrix1[i][t] * matrix2[t][j];
			result[i][j] = buf;
		}
	}
	\end{lstlisting}
\end{enumerate}
\section{Проведение тестирования}
\label{sec:tests}
Изначально было проведено тестирование стандартного алгоритма умножения матриц на квадратных матрицах размерами $0\times0,\,1\times1,\,3\times3,\,6\times6$ и прямоугольных $1\times2\bullet 2\times1,\,2\times4\bullet4\times2$ и $3\times5\bullet5\times5$, все тесты были пройдены успешно.
\par После чего была проведена серия тестов на матрицах $A=M\times Q$ и $B=Q\times N$, где M, N, Q принимают значения от 1 до 10, независимо друг от друга, матрицы заполняются случайными значениями. Сначала выполнялось умножение стандартным алгоритмом, результат которого принимался за контрольное значение, после чего результаты алгоритма Винограда и оптимизированного алгоритма Винограда сравнивались с контрольным значением и при совпадении результат считался корректным. Все тесты были пройдены успешно.

\par\textbf{Вывод.}
\par В данном разделе были рассмотрены листинги кода реализованных алгоритмов, а также их тестирование.
%%%% mode: latex
%%%% TeX-master: "rpz"
%%%% End:
