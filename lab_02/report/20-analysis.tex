\chapter{Аналитический раздел}
\label{cha:analysis}
В данном разделе будут рассмотрено умножение матриц, используя стандартный алгоритм и алгоритм Винограда.
\par\textbf{Стандартный алгоритм умножения матриц.}
\par Матрицей $A$ размера $m\times n$ называется прямоугольная таблица чисел, функций или алгебраических выражений, содержащая $m$ строк и $n$ столбцов. \cite{Belousov} Умножение матриц возможно, только когда число столбцов первой матрицы равно числу строк второй.
\par Таким образом при умножении матрицы $A$ размерности $m \times n$ на матрицу $B$ размерности $n \times k$ будет матрица $C$ размерностью $m \times k$, где
\begin{equation}
	c_{ij} = \sum\limits_{r=1}^{n}a_{ir}b_{rj}
\end{equation}


\par\textbf{Алгоритм Винограда.}
\par При рассмотрении результата умножения матриц видно, что каждый элемент -- скалярное произведение векторов, представляющих строки и столбцы матриц. \cite{winograd}
\par Рассмотрим 2 вектора $V = (v1, v2, v3, v4)$ и $W =(w1,w2,w3,w4)$. Их скалярное произведение равно:
\begin{equation}
V\bullet W=v1w1+w2w2+v3w3+v4w4
\end{equation}
Или:
\begin{equation}
V\bullet W=(v1+w2)(v2+w1)+(v3+w4)(v4+w3)-v1v2-v3v4-w1w2-w3w4
\end{equation}
\par Можно заметить, что выражение в правой части последнего равенства допускает предварительную обработку: его части можно вычислить заранее и запомнить для каждой строки первой матрицы и для каждого столбца второй. На практике это означает, что над предварительно обработанными элементами нам придется выполнять лишь первые два умножения и последующие пять сложений, а также дополнительно два сложения.
\par\textbf{Вывод.} 
\par Были рассмотрены стандартный алгоритм умножения матриц и алгоритм Винограда, который позволяет предварительно рассчитать некоторые значения, в результате чего уменьшается доля умножений.