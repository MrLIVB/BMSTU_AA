\Introduction
Расстояние Левенштейна -- это минимальное количество редакторских операций, которые необходимы для превращения одной строки в другую.
Оно может применяться для решения следующих задач:
\begin{itemize}
	\item исправления ошибок в слове;
	\item предложение вариантов поиска в поисковой строке;
	\item в биоинформатике для сравнения белков.
\end{itemize}
Целью данной лабораторной работы является реализация и сравнение алгоритмов поиска расстояний Левенштейна и Дамерау-Левенштейна.
При выполнении лабораторной работы поставлены такие задачи:
\begin{enumerate}[1)]
	\item дать математическое описание расстояния Левенштейна;
	\item описать алгоритм поиска редакторского расстояния;
	\item оценить затраты памяти на выполнение алгоритмов;
	\item провести замеры процессорного времени работы на серии экспериментов;
	\item провести сравнительный анализ алгоритмов.
\end{enumerate}